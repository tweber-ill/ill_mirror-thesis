%
% project plan / abstract
% @author Tobias Weber <tweber@ill.fr>
% @date jan-2021
% @license see 'LICENSE' file
%

\chapter*{Abstract}
\addcontentsline{toc}{chapter}{Abstract}

The triple-axis spectrometer (TAS) \cite{Shirane2002} was invented by Bertram Brockhouse in the 1950s and
is among the foremost measurement methods in the field of inelastic neutron scattering. 
It enables the probing of vibrational (phonon) or magnetic (magnon) excitations in single-crystals and allows 
mappings of their dispersion relations, i.e. their energy-momentum relation, $E\left( \underline{Q} \right)$.

The three axes of a TAS are offset by relative angles to one another, and comprise 
(i) the reactor-monochromator-sample axis, where a specific neutron energy is picked out of the polychromatic 
beam coming from the reactor's moderator; 
(ii) the monochromator-sample-analyser axis, whose angle selects a specific momentum transfer, $\underline{Q}$, 
from the neutron to the sample; and 
(iii) the sample-analyser-detector axis, which selects the energy transfer, $E$.

During the usual operation of a TAS, the user selects $\left( \underline{Q}, E \right)$ coordinates
in the reciprocal (dual) crystal space of the sample to be measured. While the vector space of crystal coordinates
is in general non-Euclidean, crystal coordinates have a one-to-one correspondence with the axis angles 
of the TAS. The correspondence can be calculated by the so-called ``$UB$ matrix formalism'' \cite{Lumsden2005}. 
Here, $B$ is the transformation matrix from crystal to lab coordinates and $U$ is a rotation to a specific 
crystal plane. From that, the TAS angles can be derived using Bragg's law.

Due to angular constraints by cables and tubes, as well as spatial constraints from the cramped instrument space,
not every $\left( \underline{Q}, E \right)$ coordinate point is accessible, and a careful mapping of each point is
usually required beforehand to avoid collisions of the instrument with walls, collisions with itself or movements
which could pull out fragile cables.

The goal this work is the development and implementation of a pathfinding algorithm for
neutron triple-axis spectrometers.
Given a user-selected target coordinate in reciprocal crystal space,
the algorithm is able to navigate the instrument in its constrained angular space.
To that end, a generalised Voronoi diagram is used to calculate a mesh of possible paths
the instrument can move along. The specific path to be used can be parametrised given
user-defined modifiers like the shortest path or the greatest distance to walls.
The situation is similar to moving a robot arm around obstacles, with the addition of having
the start and target coordinates in a coordinate system with a different metric.
An easy to use graphical user interface is provided for visualising the motion planning
problem and its solutions.
