%
% notation
% @author Tobias Weber <tweber@ill.fr>
% @date mar-2021
% @license see 'LICENSE' file
%

\chapter{Notation}
\label{ch:notation}

\section{Mathematical symbols}
Vectors in this work are written either as small letters with underscore, e.g. $\underline{v}$, 
in component notation, e.g. $\left( v^i \right)$, or in Dirac notation \cite{wiki_braket}, which is 
very popular in physics, e.g. $\left| v \right>$. 
The first notation is chosen in cases when co- or contravariance \cite[p. 806]{Arens2015} does not play a role,
the latter two, when it does. 
Contravariant vectors in component notation are written with a superscript (this is not a power!), 
covariant vectors with a subscript.
Also for the component notation, we furthermore use the Einstein summation convention \cite{wiki_summation} \cite[p. 804]{Arens2015},
meaning that a sum is implied over indices which appear twice, one in a contravariant, and one
in a covariant position, e.g.:
\begin{equation}
	\sum_{i=1}^{n} u_i v^i \ \equiv \ u_i v^i.
\end{equation}
Table \ref{tab:notation} gives details on this notation.

\begin{table}[htb]
	\centering
	\begin{tabular}{|c|c|}
		\hline
		\bf{Notation} & \bf{Explanation} \tabularnewline
		\hline
		$ \underline{v} $ & A vector (either contravariant or covariant). \tabularnewline
		\hline
		$ M \,=\, \left( m_{ij} \right) $ & A matrix $M$ with elements $m_{ij}$. \tabularnewline
		\hline
		$ M^t \,=\, \left( m_{ji} \right) $ & Transpose of matrix $M$. \tabularnewline
		\hline
		$ M^{-1} $ & Inverse of matrix $M$. \tabularnewline
		\hline
		$\left| x \right> \,=\, \left( x^i \right) $ & A contravariant vector with superscript index \cite[pp. 806]{Arens2015}. \tabularnewline
		\hline
		$\left< x \right| \,=\, \left( x_i \right) $ & A covariant vector with subscript index \cite[pp. 806]{Arens2015}. \tabularnewline
		\hline
		$s \,=\, \left< x | y \right> \,=\, x_i y^i \,=\, g_{ij} x^i y^j $ & A scalar/inner product \cite[pp. 808]{Arens2015}, \tabularnewline
			& the sums over $i$ and $j$ are implied \cite{wiki_summation}. \tabularnewline
		\hline
		$\left(a^{i}_{\;j}\right) \,=\, \left| x \right> \left< y \right| \,=\, x^i y_j$ & A tensor/outer product \cite[pp. 810]{Arens2015}. \tabularnewline
		\hline
	\end{tabular}
	\caption[Symbol notation.]{Notation of used symbols.}
	\label{tab:notation}
\end{table}

\newpage
\section{Abbreviations}
Table \ref{tab:abbreviations} lists abbreviations that are used in this work.

\begin{table}[htb]
	\centering
	\begin{tabular}{|c|c|}
		\hline
		\bf{Abbreviation} & \bf{Explanation} \tabularnewline
		\hline
		API		 & \underline{A}pplication \underline{P}rogramming \underline{I}nterface. \tabularnewline
		\hline
		APSP		 & \underline{A}ll \underline{P}airs \underline{S}hortest \underline{P}ath \cite[pp. 309-320]{Erickson2019}. \tabularnewline
		\hline
		CAD              & \underline{C}omputer \underline{A}ided \underline{D}esign. \tabularnewline
		\hline
		CGAL             & \underline{C}omputational \underline{G}eometry \underline{A}lgorithms \underline{L}ibrary \cite{web_cgal}. \tabularnewline
		\hline
		CLI              & \underline{C}ommand \underline{L}ine \underline{I}nterface. \tabularnewline
		\hline
		CNC              & \underline{C}omputer \underline{N}umerical \underline{C}ontrol \cite{wiki_milling}. \tabularnewline
		\hline
		DOI              & \underline{D}igital \underline{O}bject \underline{I}dentifier. \tabularnewline
		\hline
		GCC              & \underline{G}NU \underline{C}ompiler \underline{C}ollection \cite{web_gcc}. \tabularnewline
		\hline
		GPU              & \underline{G}raphics \underline{P}rocessing \underline{U}nit. \tabularnewline
		\hline
		GUI              & \underline{G}raphical \underline{U}ser \underline{I}nterface. \tabularnewline
		\hline
		ILL              & \underline{I}nstitut \underline{L}aue-\underline{L}angevin, Grenoble, France \cite{web_ill}. \tabularnewline
		\hline
		RLU              & \underline{R}elative \underline{L}attice \underline{U}nits, see chapter \ref{ch:xtal}. \tabularnewline
		\hline
		SO(n)            & \underline{S}pecial \underline{O}rthogonal Group in \underline{n} dimensions \cite[pp. 849-851]{Arfken2013}. \tabularnewline
		\hline
		SSSP		 & \underline{S}ingle \underline{S}ource \underline{S}hortest \underline{P}ath \cite[pp. 273-297]{Erickson2019}. \tabularnewline
		\hline
		TAS              & \underline{T}riple \underline{A}xis \underline{S}pectrometer \cite{Shirane2002}. \tabularnewline
		\hline
		UB method        & Not technically an abbreviation, but the product of two matrices \tabularnewline
		                          & which also give the name to the standard method for crystalographic \tabularnewline
		                          & coordinate calculation in spectrometers \cite{Lumsden2005}. \tabularnewline
		\hline
	\end{tabular}
	\caption[Abbreviations.]{Abbreviations used in this work.}
	\label{tab:abbreviations}
\end{table}
