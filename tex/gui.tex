%
% path-finding user interfaces
% @author Tobias Weber <tweber@ill.fr>
% @date july-2021
% @license see 'LICENSE' file
%

\chapter{User Interfaces}
\label{ch:gui}

The software has been realised in a modular fashion, it comprises a core
and a library module (see chapter \ref{ch:impl}), which can be easily linked 
into external C++ applications. Such a library usage will be important in the
future when we plan to utilise the software as a plug-in module to the instrument
control software \textit{NOMAD} \cite{web_NOMAD} that is employed at the Institut 
Laue-Langevin.

Furthermore, a graphical user interface (GUI) has been implemented for a visual
and interactive representation of the instrument and the underlying algorithms.
The GUI is described in detail in section \ref{sec:gui}.

Finally, the software allows scripting via \textit{Python} \cite{Rossum2011, web_python}. 
Apart from simply setting up a workflow and plotting the results, the \textit{Python}
interface will allow the usage of the software in \textit{Python}-based instrument
control systems such as \textit{NICOS} \cite{web_NICOS}, which is used at the 
Forschungsreaktor M\"unchen II (FRM-II). 
Details on the scripting interface can be found in section \ref{sec:scripting}.




\section{Graphical User Interface}
\label{sec:gui}

The software's main graphical user interface (GUI), for which a typical session is depicted
in Fig. \ref{fig:gui}, is based on the \textit{Qt} framework \cite{web_Qt}, which allows
for an easy and rapid cross-platform GUI development in \textit{C++}. We support both
current releases of \textit{Qt}, namely version 5 and version 6.
Similar to the core calculation module, the GUI is written in the recent \textit{C++20}
standard \cite{ISOCPP20} of the C++ language family \cite{Stroustrup2008, Stroustrup2018}.
The source code of the GUI module can be found in the directory \lstinline|./src/gui| of the
source repository, see chapter \ref{ch:online} for more information.

\begin{figure}[htb]
		\begin{center}
			\includegraphics[width = 1 \textwidth]{figures/gui}
		\end{center}
	\caption{Main GUI. Here, instrument and sample crystal properties can be set up,
		walls can be added and moved and paths around them be calculated.
		The central view provides a three-dimensional visualisation of the instrument
		configuration and is fully dynamic: Every element, including the instrument
		and the wall segments, can be moved or manipulated using the mouse.
		\label{fig:gui}}
\end{figure}

The GUI allows the set-up of an instrument configuration and a crystal $UB$ matrix from a
sample definition according to the formalism described in chapter \ref{ch:xtal}.
All calculation is performed in the core module, which is described in chapter \ref{ch:impl}
and to which the GUI is just that, an interface. By the same token, the core module itself is
completely independent of the GUI, or any other interface code, and the full functionality
of the software, except GUI-specific visualisations, is equally accessible from the other
alternative interface modules, e.g. the \textit{Python} interface described in section
\ref{sec:scripting}, or the raw \textit{C++} library interface. Graphical display of the
instrument and the walls as well as the interaction with these elements is performed
using \textit{OpenGL} \cite{web_OpenGL} via Qt's \lstinline[language=C++]|QOpenGLWidget|
\cite{web_QOpenGLWidget} class.



\begin{figure}[htb]
		\begin{center}
			\includegraphics[width = 0.66 \textwidth]{figures/gui_configspace}
		\end{center}
	\caption{Angular configuration space and path calculation. The figure plots all
	possible instrument positions for the monochromator and sample scattering angles,
	$2\theta_M$ and $2\theta_S$, respectively. Forbidden positions are shown in red.
	These can be invalid angles as well as collisions of the instrument with walls
	(here, specifically, the pillar from Fig \ref{fig:gui}), or with itself.
	Allowed positions are drawn in blue. The mesh of all possible instrument
	paths is shown as white lines, while a currently selected example path from
	the red start to the green target position is shown as a yellow line.
		\label{fig:gui_configspace}}
\end{figure}




\section{Python Scripting Interface}
\label{sec:scripting}
