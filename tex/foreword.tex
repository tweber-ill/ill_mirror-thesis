%
% foreword
% @author Tobias Weber <tweber@ill.fr>
% @date sep-2021
% @license see 'LICENSE' file
%

\chapter*{Foreword}
\addcontentsline{toc}{chapter}{Foreword}
The present work is concerned about the development of a pathfinding system for neutron triple-axis
spectrometers using the concepts of robot motion planning.
It touches upon multiple disciplines in physics and computer science. Among these are the field
of neutron scattering, concepts of solid-state physics and crystallography for the physical part,
as well as Voronoi diagrams, graph theory and spatial index trees for robot motion 
planning, furthermore three-dimensional graphics and unit testing.
This work is organised in a manner to introduce important concepts with dedicated chapters that are
as self-contained as possible. These introductory chapters appear directly before the chapters that reference
them.

\paragraph{Layout of this work}
After some general introduction in chapter \ref{ch:intro}, the two subsequent chapters 
provide the necessary building blocks for the development of the pathfinding algorithm that follows.
These are chapter \ref{ch:xtal}, which reviews important concepts from crystallography and
neutron scattering, namely crystal and instrument coordinate systems,
as well as chapter \ref{ch:algos}, which presents data structures and algorithms 
that are necessary for the pathfinding.
Chapter \ref{ch:paths} is concerned about the general concept of pathfinding and motion planning, providing
a review of different alternatives for the present problem as well as a high-level overview of the algorithm
to be developed in this work. The actual details and the implementation of the pathfinding algorithm
is outlined in chapter \ref{ch:impl}.
Chapter \ref{ch:gl} contains a brief introduction to the mathematics of three-dimensional computer graphics.
This mathematical formalism is made use of in chapter \ref{ch:gui} which is concerned about the user interfaces to the software
developed in this work. The user interfaces include a graphical one based on an interactive, three-dimensional
view of the instrument, as well as a purely command-line scripting interface.
Chapter \ref{ch:tests} describes unit testing necessary for maintaining
a reasonably sized software relatively bug-free.
The closing chapter, \ref{ch:outlook}, discusses the results and gives an outlook 
on future developments.

\paragraph{Accompanying software}
The source code of the software that is part of this work is provided on a supplementary 
medium together with compiled binary versions for several systems.
The source code is furthermore available via the DOI (digital object identifier) 
\href{https://doi.org/10.5281/zenodo.4625649}{10.5281/zenodo.4625649} and can be
accessed online via the URL \url{https://doi.org/10.5281/zenodo.4625649}.
An up-to-date list of possible future errata to this work will be provided under the DOI
\href{https://doi.org/10.5281/zenodo.5092159}{10.5281/zenodo.5092159} and the corresponding URL
\url{https://doi.org/10.5281/zenodo.5092159},
see appendix \ref{ch:online} for more details.
