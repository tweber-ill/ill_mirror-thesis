%
% outlook
% @author Tobias Weber <tweber@ill.fr>
% @date aug-2021
% @license see 'LICENSE' file
%

\chapter{Outlook and Future Developments}
\label{ch:outlook}
The software of this work is capable of finding a path around obstacles given start and target coordinates 
either in the crystal coordinate system or directly using the instrument angles.
While the software is fully functional, several more steps will need to be done to make it usable
at the instrument.

The first step is an integration into existing instrument control systems. 
With its modular design and the possibility to use it as C++ library without any graphical user interface
(see chapter \ref{sec:library}), care has already been taken to make this step as smooth as possible.
In a joint project with Y. Le Goc from the instrument control group at the ILL, a plug-in module for
the instrument control software \textit{NOMAD} \cite{web_NOMAD} will be developed that links against 
the path-finding library of this work. 
The idea is to have a settings tab in \textit{NOMAD} where the user can switch on or off the path-finding
functionality. If it is on, subsequent drive commands will not be directly executed and all motors thus
driven at the same time independently, but instead piped through the plug-in module. 
The module will give back a list of drive commands to \textit{NOMAD} which drives the instrument
along the determined optimal path in a step-wise fashion.

A second step will include an interface with the independent \textit{NOMAD 3D} \cite{web_NOMAD3d}
software to replace the collision detection performed on the simplified model with the full \textit{CAD}
model \cite{ThalesModel2021} of the instrument.
So far, \textit{NOMAD 3D} tries to calculate collisions between all polygons of the \textit{CAD} model, 
but -- as expected -- does not come to a result before running out of memory and time.
This problem of \textit{NOMAD 3D} might never be solved. 
But in that case, we have shown that a simplified model as is used in the present software is much more 
feasible. 
The simplified model may be thought of as the bounding boxes and cylinders of the real full-detail model.
The results are still correct and just rationalise away all details close to obstacles, which should
anyway be avoided.

Another development for the future that was suggested by M. B\"ohm, the leader of the spectroscopy 
group at the ILL, might include the use of a depth camera for automatic generation and update 
of the obstacle positions. The camera would be at a fixed position above the instrument
space, where it has full coverage. The depth values would be used to identify any objects above the
instrument floor. Such a strategy has already been tested in the field of robot motion planning and 
algorithms for identifying objects are available \cite{Biswas2012}.
Such a feature might be useful to include non-fixed geometry such as helium cans or ladders
that may be present in the instrument space.

A first general presentation of the software of this work will take place at the 
\textit{Innovative Inelastic Neutron Scattering} workshop in Autrans-M\'eaudre-en-Vercors, France 
in October 2021.
