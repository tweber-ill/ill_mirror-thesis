%
% outlook
% @author Tobias Weber <tweber@ill.fr>
% @date aug-2021
% @license see 'LICENSE' file
%

\chapter{Discussion and Outlook}
\label{ch:outlook}
The software of this work is capable of moving a triple-axis spectrometer around obstacles 
given start and target coordinates either in the crystal coordinate system or directly using the instrument angles.
In the scope of this work, a triple-axis spectrometer can be thought of as a robot arm having three rotational
degrees of freedom and moving on a planar floor.
The program can be used as a library, via a scripting interface, and -- more comfortably -- through a graphical
user interface, which allows for a convenient editing of the geometry and testing of configuration
with immediate feedback of the results.

Ideas from the field of computational geometry proved indispensable for the success of the project,
first and foremost the concept of the Voronoi diagram and its bisectors, but also index trees and sweep 
algorithms, which ensure the efficiency of the pathfinding algorithm.
The calculation of Voronoi diagrams for line segments which approximate curves is also the most
calculation-intensive part of the algorithm, together with the quantisation of the configuration space.
In total, our approach to pathfinding is a mix between what Choset calls the grid-based model and the
topological approach \cite{Choset2010_ch8}.
The purely topological approach would consist of calculating the form of the obstacles analytically in
configuration space and computing from that the medial axis (see chapter \ref{sec:voro_median}), also
in analytical fashion. 
Analytical obstacle calculation is too complex mathematically due to the coordinate systems and trigonometric
expressions involved (see chapter \ref{ch:xtal}). Analytical calculation of the medial axis, which would consist
of calculating the Voronoi bisectors of curved line segments in a closed form, is also not feasible, 
as was discussed in chapter \ref{sec:voro_median}.
Our quantisation step therefore mainly served to reduce the problem of the medial axis to the standard
line segment Voronoi diagrams, for which very efficient libraries are available. 
A purely grid-based approach, on the other hand, would not keep the instrument at the furthest distance
of the obstacles.

The importance of a pathfinding system for triple-axis spectrometers can be seen as the development
of a software with a similar goal had already been attempted before, in a tool named 
\textit{Paprika} \cite{Muehlbauer2006}, a fact that was brought to our attention at the very end of the present work.
At its time, \textit{Paprika} gained relatively little traction, though, and this might be due to their approach 
being less than optimal.
Similar to our general approach, they also construct a two-dimensional configuration space of allowed collision-free 
angles of the instrument on a grid. But instead of using it to build a path mesh of Voronoi/medial axis 
bisectors along which the instrument can move at the furthest distance of any obstacle, which is central 
to our approach, they directly use a grid-based method, finding the shortest path directly among the grid vertices.
This can lead to the shortest path leading directly along a wall.
As the \textit{Paprika} project dates from 2006, a reason why they did not consider Voronoi bisectors 
for possible path curves might be the difficulty -- both conceptually and in terms of processor time -- 
of calculating them for curved line segments, even if they are -- as in our case -- quantised into straight 
line segments.
Our performance measurements depicted in Fig. \ref{fig:voro_performance} of chapter \ref{sec:voronoi_libs}
show that on a modern processor of 2021 typical calculation times are in the order of tens of seconds.
15 years ago, the same calculations might still have been out of reach for practical applications.


\paragraph{Future developments}
While the present software is already fully functional, several more steps will need to be done 
to make it usable at the instrument.

The first step is an integration into existing instrument control systems. 
With its modular design and the possibility to use it as C++ library without any graphical user interface
(see chapter \ref{sec:library}), care has already been taken to make this step as smooth as possible.
In a joint project with Y. Le Goc from the instrument control group at the ILL, a plug-in module for
the instrument control software \textit{NOMAD} \cite{web_NOMAD} will be developed that links against 
the pathfinding library of this work.
The idea is to have a settings tab in \textit{NOMAD} where the user can switch on or off the pathfinding
functionality. If it is on, subsequent drive commands will not be directly executed and all motors thus
driven at the same time independently, but instead piped through the plug-in module. 
The module will give back a list of drive commands to \textit{NOMAD} which drives the instrument
along the determined optimal path in a step-wise fashion.

A second step will include an interface with the independent \textit{NOMAD 3D} \cite{web_NOMAD3d}
software to replace the collision detection performed on the simplified model with the full \textit{CAD}
model \cite{ThalesModel2021} of the instrument.
So far, \textit{NOMAD 3D} tries to calculate collisions between all polygons of the \textit{CAD} model, 
but -- as expected -- does not come to a result before running out of memory and time.
This problem of \textit{NOMAD 3D} might never be solved. 
But in that case, we have shown that a simplified model as is used in the present software is much more 
feasible. 
The simplified model may be thought of as the bounding boxes and cylinders of the real full-detail model.
The results are still correct and just rationalise away all details close to obstacles, which should
anyway be avoided.

Another development for the future that was suggested by M. B\"ohm, the leader of the spectroscopy 
group at the ILL, might include the use of a depth camera for automatic generation and update 
of the obstacle positions. The camera would be at a fixed position above the instrument
space, where it has full coverage. The depth values would be used to identify any objects above the
instrument floor. Such a strategy has already been tested in the field of robot motion planning and 
algorithms for identifying objects are available \cite{Biswas2012}.
Such a feature might be useful to include non-fixed geometry such as helium cans or ladders
that may be present in the instrument space.

A first general presentation of the software of this work will take place at the 
\textit{Innovative Inelastic Neutron Scattering} workshop in Autrans-M\'eaudre-en-Vercors, France 
in October 2021.
