%
% gl
% @author Tobias Weber <tweber@ill.fr>
% @date aug-2021
% @license see 'LICENSE' file
%

\chapter{3-d Rendering and OpenGL}
\label{ch:gl}

The software of this work uses \textit{OpenGL} \cite{web_OpenGL} as its main graphical user interface (GUI).
This appendix chapter adds technical details to the general descriptions in chapter \ref{sec:gui_gl}.



\section{Coordinate Systems and Transformations}

To draw a three-dimensional object onto the screen, three principal coordinate systems are typically maintained,
namely the coordinate system of the object, the system of the camera, and screen coordinates.
The vertices of the 3-d object, $\left|x\right>$ , are thus transformed onto the screen by using three
transformation matrices,
\begin{equation}
	\left|x_{\mathrm{screen}}\right> \ =\ P \cdot V \cdot  M \cdot  \left| x \right>,
	\label{eq:gl_mvp}
\end{equation}
where $M$ represents the model's local matrix (used, for example, to spin the object around one of its axes),
$V$ is the view matrix given by the camera's coordinate system which transforms local object coordinates into
a global coordinate system, and $P$ finally projects the three-dimensional scene onto the two-dimensional plane
representing the screen.

Historically, \textit{OpenGL} used a fixed transformation and lighting pipeline \cite{wiki_gl_history} for
these transformations. The pipeline maintained different stacks of matrices for the combined model-view and
projection matrices \cite{web_gl_matrixmode}.
Modern \textit{OpenGL} versions do not use internal matrix stacks anymore, instead all transformation
and lighting operations are programmable through small \textit{shader} programs that are directly executed on
the graphics processing unit (GPU) \cite{wiki_gl_history}. These programs are written in the
\textit{OpenGL Shading Language} (\textit{GLSL}), which belongs to the \textit{C} family of languages, but
only uses a subset of \textit{C} \cite{wiki_glsl}.
While the transformation given by Eq. \ref{eq:gl_mvp} is not fixed anymore, its form is usually maintained
in the shaders by the user.



\subsection{3-d transformations}



\subsection{3-d to 2-d projections}
