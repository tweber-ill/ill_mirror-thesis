%
% gl
% @author Tobias Weber <tweber@ill.fr>
% @date aug-2021
% @license see 'LICENSE' file
%

\chapter{Interlude: Mathematics for 3-d Graphics}
\label{ch:gl}

The software of this work uses \textit{OpenGL} \cite{web_OpenGL} in its main graphical user interface (GUI).
Before discussing the main GUI proper, this chapter reviews concepts behind 3-d graphics in a general way.
Here, we mainly describe the most important mathematical aspects of 3-d graphics, a comprehensive overview of 
\textit{OpenGL} and the mathematics involved can be found in Ref. \cite{Sellers2002}.



\section{Coordinate systems and transformations}
To draw a three-dimensional object onto the screen, several principal coordinate systems are typically maintained,
namely the coordinate system of the 3-d object, a global coordinate system common for all objects in scene, the
system of the camera, as well as screen (device) and window coordinates \cite[pp. 63-66]{Sellers2002}.
The vertices of the 3-d object, $\left|x\right>$ , are thus transformed into a window's 2-d pixels by using four
matrices transforming from one coordinate system to the next,
\begin{equation}
	\boxed{\left|x_{\mathrm{window}}\right> \ =\ W \cdot P \cdot V \cdot  M \cdot  \left| x_{\mathrm{model}} \right>,}
	\label{eq:gl_mvp}
\end{equation}
where $M$ is a matrix transforming the model's local system (used, for example, to spin the object around one of
its axes) into the global coordinate system, $V$ is the view matrix given by the camera's coordinate system
which transforms global object coordinates into the camera's viewing frame. These matrices take the form
described in section \ref{sec:gl_trafos}.
$P$ projects the three-dimensional scene onto the two-dimensional plane representing the screen \cite{web_gl_ortho, web_gl_perspective} and is described in section \ref{sec:gl_projs}. 
Finally, $W$ is the window or viewport matrix which scales and translates the projected screen coordinates 
into the pixel coordinates of the display window \cite{web_gl_viewport}, it is discussed in section \ref{sec:gl_viewport}.

Historically, \textit{OpenGL} used a fixed transformation and lighting pipeline \cite{wiki_gl_history} for
these transformations. The pipeline maintained different stacks of matrices for the combined model-view and
projection matrices \cite{web_gl_matrixmode}.
Modern \textit{OpenGL} \cite{web_OpenGL} versions do not use internal matrix stacks anymore, instead all
transformation and lighting operations of the rendering pipeline are programmable through small \textit{shader}
programs that are directly executed on the graphics processing unit (GPU) \cite{wiki_gl_history}
and are written in a \textit{C}-like language \cite{wiki_glsl}.
Going even further than \textit{OpenGL}, its successor \textit{Vulkan} \cite{web_Vulkan} makes the pipeline
itself dynamic and programmable, not only the shaders of its individual steps.
While the transformation given by Eq. \ref{eq:gl_mvp} is not fixed anymore, its form is usually maintained
in the shaders by the user. More information on the rendering pipeline and the shaders is given in 
section \ref{sec:gl_shaders}.

The transformation inverse to Eq. \ref{eq:gl_mvp} is of importance to be able to select objects in the 
three-dimensional scene by mouse. 
To that end, a line is calculated by ``unprojecting'' two homogeneous $\left|x_{\mathrm{window}}\right>$ vectors
(see next section) having the same mouse position as their $x$ and $y$ components and the distance of the
near and far planes of the view frustum as two different $z$ coordinate components.
Finally, the intersection of the line through these two unprojected $\left|x_{\mathrm{model}}\right>$ points 
with the scene geometry is calculated. 
The unprojection operation is given by \cite{web_gl_unproject}:
\begin{equation}
	\left|x_{\mathrm{model}}\right> \ =\ M^{-1} \cdot V^{-1} \cdot P^{-1} \cdot  W^{-1} \cdot  \left| x_{\mathrm{window}} \right>,
\end{equation}
and normalising $\left|x_{\mathrm{model}}\right>$ by its fourth (homogeneous) component \cite{web_gl_unproject}.



\section{3-d transformations}
\label{sec:gl_trafos}
To simplify the calculations when translations are involved, \textit{OpenGL} uses homogeneous coordinates.
Homogeneous coordinates append a fourth component to the three-dimensional vectors, which is set to zero for
directions and non-zero for vertices, where the original vertex coordinates are recovered by normalising this
fourth component to one \cite[pp. 235, 357]{Bronstein2008}.


% -----------------------------------------------------------------------------
% scaling
% -----------------------------------------------------------------------------
\subsection{Translation}
Vertices can thus be translated by left-multiplying with a matrix whose fourth column represents the translation
vector \cite{web_gl_translate}:
\begin{equation}
	\left( \begin{array}{c} x + t_x \\ y + t_y \\ z + t_z \\ 1 \end{array} \right) 
	\ =\  
	\left( \begin{array}{cccc} 
		1 & 0 & 0 & t_x \\
		0 & 1 & 0 & t_y \\
		0 & 0 & 1 & t_z \\
		0 & 0 & 0 & 1
	\end{array} \right) \cdot
	\left( \begin{array}{c} x \\ y \\ z \\ 1 \end{array} \right).
\end{equation}
Not that direction vectors, which have their fourth component zero, are not affected by a translation.
% -----------------------------------------------------------------------------


% -----------------------------------------------------------------------------
% scaling
% -----------------------------------------------------------------------------
\subsection{Scaling}
By the same token, scaling along the principal axes is performed by setting the diagonal matrix 
elements \cite{web_gl_scale}:
\begin{equation}
	\left( \begin{array}{c} s_x \cdot x \\ s_y \cdot y \\ s_z \cdot z \\ 1 \end{array} \right) 
	\ =\  
	\left( \begin{array}{cccc} 
		s_x & 0 & 0 & 0 \\
		0 & s_y & 0 & 0 \\
		0 & 0 & s_z & 0 \\
		0 & 0 & 0 & 1
	\end{array} \right) \cdot
	\left( \begin{array}{c} x \\ y \\ z \\ 1 \end{array} \right).
\end{equation}
% -----------------------------------------------------------------------------


% -----------------------------------------------------------------------------
% rotation
% -----------------------------------------------------------------------------
\subsection{Rotation}
The set of rotations in three-dimensional space forms an algebraic group, the special orthogonal group $\mathrm{SO\left(3\right)}$,
which comprises all orthogonal matrices with determinant 1.
There are two popular ways to describe rotations $R \in \mathrm{SO\left(3\right)}$ given the rotation axis and angle, 
namely the Rodrigues formalism and quaternions. 
% As they are both ubiquitous in computer graphics, we describe each of them briefly in the following paragraphs.
For clarity, we only write out the left-upper 3x3 part of the homogeneous 4x4 matrices.


\subsubsection{Rodrigues formalism}
For the derivation in this section we follow Refs. \cite[p. 718, p. 816]{Arens2015} and \cite{wiki_rodrigues}.
The matrix $R$ describing the rotation about a normalised vector $\left|v\right>$ under an angle $\alpha$ can be decomposed into
three projections onto the rotated coordinate system \cite[p. 718, p. 816]{Arens2015}:
\begin{equation}
	\boxed{R \ = \ P_{||} + P_{\perp} \cos \alpha + P_{\times} \sin \alpha.}
	\label{eq:rodrigues}
\end{equation}
Here, $P_{||}$ is the projector onto the axis itself, which remains invariant under a rotation and is written as \cite[p. 814]{Arens2015}:
\begin{equation}
	P_{||} \ =\ \left|v\right> \left<v\right| \ =\ 
		\left( \begin{array}{ccc} 
			v_1^2   &    v_1 v_2 &       v_1 v_3 \\
			v_2 v_1 &      v_2^2 &       v_2 v_3 \\
			v_3 v_1 &    v_3 v_2 &         v_3^2
		\end{array} \right).
\end{equation}
To see that this is the case, one can apply a the projector to a test vector $\left| x \right>$ resulting in $\left|v\right> \left<v | x \right>$,
which is the projection of $\left| x \right>$ onto $\left| v \right>$ in the direction of $\left| v \right>$ \cite[p. 814]{Arens2015}.

The second axis for rotation is found by applying $P_{\perp}$, the orthogonal projector onto the plane, 
whose normal is given by $\left|v\right>$, and which reads \cite[p. 814]{Arens2015}
\begin{equation}
	P_{\perp} \ =\ 1 - \left|v\right> \left<v\right| \ =\ 
		\left( \begin{array}{ccc} 
			1 - v_1^2    &      - v_1 v_2 &         - v_1 v_3 \\
			   - v_2 v_1 &      1 - v_2^2 &         - v_2 v_3 \\
			   - v_3 v_1 &      - v_3 v_2 &         1 - v_3^2
		\end{array} \right).
\end{equation}
Here, $1$ names the unit matrix, and the equation is derived from completeness relation of the vector space under
a basis $\left| v_i \right>$ \cite[p. 814]{Arens2015}:
\begin{equation}
	1 \ =\  \sum_i \left| v_i \right> \left< v_i \right|.
\end{equation}

Finally, the third axis is found by applying $P_{\times}$, the skew-symmetric matrix for $\left|v\right>$.
This is the matrix which -- when applied to a vector $\left|x\right>$ -- gives the cross product of $\left|v\right>$ and $\left|x\right>$, 
and it reads \cite{wiki_skewsymm}:
\begin{equation}
	P_{\times} \ =\ 
		\left( \begin{array}{ccc} 
			0     & -v_3 &  v_2 \\
			v_3   & 0    & -v_1 \\
			-v_2  & v_1  & 0
		\end{array} \right).
\end{equation}

Explicitly writing Eq. \ref{eq:rodrigues} thus yields the full Rodrigues formula \cite[p. 718, p. 816]{Arens2015}:
\begin{equation}
	R \ = \ 
		\left( \begin{array}{ccc} 
			v_1^2   &    v_1 v_2 &       v_1 v_3 \\
			v_2 v_1 &      v_2^2 &       v_2 v_3 \\
			v_3 v_1 &    v_3 v_2 &         v_3^2
		\end{array} \right)
	+ 	\left( \begin{array}{ccc} 
			1 - v_1^2    &      - v_1 v_2 &         - v_1 v_3 \\
			   - v_2 v_1 &      1 - v_2^2 &         - v_2 v_3 \\
			   - v_3 v_1 &      - v_3 v_2 &         1 - v_3^2
		\end{array} \right)
		\cos \alpha 
	 + 
		\left( \begin{array}{ccc} 
			0     & -v_3 &  v_2 \\
			v_3   & 0    & -v_1 \\
			-v_2  & v_1  & 0
		\end{array} \right)
		\sin \alpha,
	\label{eq:rodrigues_expl}
\end{equation}
which is, for example, used in \textit{OpenGL}'s  \lstinline[language=C]|glRotate| functions \cite{web_gl_rotate}.

Using the canonical coordinate basis vectors $\left| x \right> = \left( 1\,0\,0 \right)^t$, $\left| y \right> = \left( 0\,1\,0 \right)^t$,
and $\left| z \right> = \left( 0\,0\,1 \right)^t$, respectively, we reproduce the well-known simple rotation matrices as 
special cases \cite[p. 238]{Bronstein2008}:
\begin{equation}
	R_x \ =\ 
		\left( \begin{array}{ccc} 
			1 &            0 &            0  \\
			0 &  \cos \alpha & -\sin \alpha  \\
			0 & \sin \alpha  &  \cos \alpha
		\end{array} \right),\,
	R_y \ =\ 
		\left( \begin{array}{ccc} 
			  \cos \alpha &         0 &  \sin \alpha \\
			           0  &         1 &            0 \\
			-\sin \alpha  &         0 &  \cos \alpha
		\end{array} \right),\,
	R_z \ =\ 
		\left( \begin{array}{ccc} 
			 \cos \alpha & -\sin \alpha &             0  \\
			\sin \alpha  &  \cos \alpha &             0  \\
			           0 &            0 &             1
		\end{array} \right).
\end{equation}


\subsubsection{Quaternions}
Quaternions are hyper-complex numbers of the form $w + x \cdot i + y \cdot j + z \cdot k$, possessing one real 
component, $w$, and three imaginary components, $x$, $y$, and $z$. 
Usually, quaternions are written as tuples $\left( w,\,x,\,y,\,z \right) = \left( w,\,\underline{v} \right) \in \mathbb{H}$,
where the three imaginary components are treated as a vector component.
Algebraically, quaternions form the division ring $\mathbb{H}$, named after William Rowan Hamilton, which follows from the 
basic properties of the three imaginary units, $i$, $j$, and $k$ \cite[p. 103]{Kuipers2002}:
\begin{equation}
	i^2 \ =\ j^2 \ =\ k^2 \ =\ ijk \ =\ -1.
	\label{eq:quat_basic}
\end{equation}
A very good in-depth treatise of quaternion algebra and analysis can be found in the book by Kuipers \cite{Kuipers2002},
which we follow in this section.
Here we only present some important results, please refer to \cite{Kuipers2002} for the derivations.

\paragraph{Quaternion multiplication}
The product of two quaternions $\left( w_1,\,\underline{v_1} \right),\ \left( w_2,\,\underline{v_2} \right) \in \mathbb{H}$ 
derives from the basic algebraic property (Eq. \ref{eq:quat_basic}) and is \cite[pp. 106-110]{Kuipers2002}:
\begin{equation}
	\left( w_1, \,\underline{v_1} \right) \cdot \left( w_2, \;\underline{v_2} \right) \ \equiv \ 
	\left( w_1 w_2 - \underline{v_1} \cdot \underline{v_2}, \;\underline{v_1} \times \underline{v_2} + w_1 \underline{v_2} + w_2 \underline{v_1}\right).
\end{equation}
Multiplying a quaternion $q = \left( r,\, \underline{s} \right) \in \mathbb{H}$ with a vector 
$\underline{v} \in \mathbb{R}^3$  reduces to two quaternion multiplications and is done by \cite[p. 127]{Kuipers2002}:
\begin{equation}
	q \cdot \underline{v} \ =\ 
	q \cdot \left(0,\ \underline{v} \right) \cdot q^{*} \ =\ 
	\left( r,\, \underline{s} \right) \cdot \left(0,\ \underline{v} \right) \cdot \left( r,\, -\underline{s} \right).
	\label{eq:mult_quat_vec}
\end{equation}
where $q^{*} = \left( r,\, -\underline{s} \right)$ is the complex conjugate quaternion \cite[pp. 110-111]{Kuipers2002}.

\paragraph{SO(3)}
Representatnts of the special orthogonal group $\mathrm{SO\left(3\right)}$, which comprises rotations in 
three-dimensional space by an angle $\alpha$ around a normalised vector $\underline{v}$ \cite[pp. 849-851]{Arfken2013},
can be directly written in quaternion form in an analogy to Euler's formula for complex numbers \cite{wiki_quatrot}:
\begin{equation}
	\boxed{
	q_{\mathrm{rot}} \ =\ \left(\cos\left(\alpha/2 \right),\; \sin\left(\alpha/2 \right) \cdot \underline{v} \right).
	}
	\label{eq:quatrot}
\end{equation}
This is the quaternion equivalent of Eq. \ref{eq:rodrigues}. One can readily prove that Eqs. \ref{eq:rodrigues} and \ref{eq:quatrot} are equal by applying both formulas to a test vector (using the multiplication rule of Eq.
\ref{eq:mult_quat_vec} in the  latter case), simplifying occurring expressions using trigonometric 
identities \cite{wiki_trig} and triple vector products of the form 
$\left(\underline{a} \times \underline{b}\right) \times \underline{c} = 
	\left( \underline{c}\cdot \underline{a} \right) \underline{b} - 
	\left( \underline{c}\cdot \underline{b} \right) \underline{a}$ \cite{wiki_tripleprod},
and verifying that the results are equal in both cases.
Note that this equivalence can also be shown more theoretically by first representing quaternions in the 
special unitary group $\mathrm{SU\left(2\right)}$, with the algebraic identity of 
Eq. \ref{eq:quat_basic} being fulfilled by a set of basis vectors comprising the three Pauli matrices and the 
identity matrix \cite[p. 116]{Arfken2013}. And, secondly, invoking the homomorphism 
between groups $\mathrm{SO\left(3\right)}$ and $\mathrm{SU\left(2\right)}$ \cite[pp. 851-852]{Arfken2013}.
% -----------------------------------------------------------------------------


% -----------------------------------------------------------------------------
% projections
% -----------------------------------------------------------------------------
\section{3-d to 2-d projections}
\label{sec:gl_projs}
To be able to display it in 2-d, the 3-d scene needs to be projected onto a plane representing the screen.
For \textit{OpenGL} these screen coordinates have to be normalised to $x, y \in \left[-1,\, 1\right]$,
as well as $z \in \left[-1,\, 1\right]$ (alternatively, $z \in \left[0,\, 1\right]$ for \textit{Vulkan}) 
\cite{web_QVulkanWindow}.

\subsubsection{Parallel projection}
The standard orthogonal projection matrix fulfilling these requirements is given as
\cite{web_gl_ortho} \cite[p. 82]{Sellers2002}:
\begin{equation}
	P_{\mathrm{ortho}} \ =\
		\left( \begin{array}{cccc}
			\frac{2}{w} &                           0 &              0 &  0                  \\
			          0 &  \sigma \cdot \frac{2}{h} &              0 &  0                  \\
			          0 &                           0 &  \frac{s}{n-f} &  \frac{n+f_0}{n-f}  \\
			          0 &                           0 &              0 &  1
		\end{array} \right),
\end{equation}
where $w$ and $h$ are the respective distances between the left and right as well as the top and bottom 
view planes, and $n$ and $f$ are the distances of the near and far planes.
$\sigma$ is a sign factor depending on the coordinate system handedness, it is left-handed for
\textit{OpenGL} and right-handed for \textit{Vulkan},
$s$ and $f_0$ are scaling factors. The values for these constants are given in table \ref{tab:gl_constants}.
As can be directly read off, the matrix consists of a scaling by
$\left(2/w, \, 2\sigma/h, \, s/(n-f) \right)^t$
followed by a translation by $\left( 0, \, 0, \, (n+f_0)/(n-f) \right)^t$,
which map the $\left[-w/2, \, w/h\right]$, $\left[-h/2, \, h/2\right]$ and $\left[n,\,f\right]$ ranges into their respective
normalised ranges, namely $\left[-1,\, 1\right]$ for \textit{OpenGL}.
A comprehensive derivation of this matrix can be found in Ref. \cite{web_webgl_ortho}.



\subsubsection{Perspective projection}
The most commonly used perspective matrix to project onto the near plane of a view frustum (truncated pyramid)
with $y$ opening angle $\phi$ is given as \cite{web_gl_perspective} \cite{web_gl_frustum} \cite[p. 81]{Sellers2002}:
\begin{equation}
	P_{\mathrm{persp}} \ =\
		\left( \begin{array}{cccc}
			\frac{2n}{w} &                          0 &                  0 &  0                 \\
			           0 &  \sigma \cdot \frac{2n}{h} &                  0 &  0                 \\
			           0 &                          0 &  \frac{n_0+f}{n-f} &  \frac{s n f}{n-f} \\
			           0 &                          0 &         \boxed{-1} &  0
		\end{array} \right) \ = \
		\left( \begin{array}{cccc}
			\frac{\rho}{\tan\left(\phi/2 \right)} &                                                0 &                  0 &  0                 \\
			                                    0 &  \sigma \cdot \frac{1}{\tan\left(\phi/2 \right)} &                  0 &  0                 \\
			                                    0 &                                                0 &  \frac{n_0+f}{n-f} &  \frac{s n f}{n-f} \\
			                                    0 &                                                0 &         \boxed{-1} &  0
		\end{array} \right),
\end{equation}
where the important difference is the $-1$ factor in row 4 and column 3, which generates the perspective
$1/z$ division upon normalisation by the fourth, homogeneous component \cite[pp. 350-351]{Kuipers2002}.
As before, the rest is merely scaling and translation to ensure that the $x$, $y$, and $z$ ranges
are fulfilled.
The variables have the same meaning as in the orthogonal projection case.
Additionally, $\phi$ names the field of view angle and $\rho$ is the screen ratio (e.g. 9/16 or 3/4).
$n_0$ is a scaling factor. Typical values are given in table \ref{tab:gl_constants}.
A full derivation of this matrix is given in Ref. \cite[pp. 350-351]{Kuipers2002}
for a simplified case, and in Ref. \cite{web_webgl_perspective} for the full
problem.


\begin{table}[htb]
	\centering
	\begin{tabular}{|c|ccc|}
		\hline
		    Constant & Remark             &  \textit{OpenGL} &   \textit{Vulkan} \tabularnewline
		\hline
		    $\sigma$ & Handedness factor  &              $1$ &              $-1$ \tabularnewline
		       $n_0$ & Near plane factor  &              $n$ &               $0$ \tabularnewline
		       $f_0$ & Far plane factor   &              $f$ &               $0$ \tabularnewline
		         $s$ & $z$ scaling        &              $2$ &               $1$ \tabularnewline
		\hline
	\end{tabular}
	\caption[Projection matrix constants]{
		Constant values for the perspective and orthogonal projection matrices used in the
		\textit{OpenGL} \cite{web_OpenGL} and \textit{Vulkan} \cite{web_Vulkan} graphics libraries.}
	\label{tab:gl_constants}
\end{table}
% -----------------------------------------------------------------------------



% -----------------------------------------------------------------------------
% window/viewport trafo
% -----------------------------------------------------------------------------
\section{Window transformation}
\label{sec:gl_viewport}
Scaling and centring of the projected screen coordinates into a graphics window with 
width $w_{\mathrm{win}}$, height $h_{\mathrm{win}}$, 
near $z$ value $n_{\mathrm{win}}$ and far $z$ value $f_{\mathrm{win}}$ is done by the 
following matrix \cite{web_gl_viewport}:
\begin{equation}
	W \ =\ 
	\left( \begin{array}{cccc} 
		\frac{1}{2} w_{\mathrm{win}} &                              0 &                                                          0 &                              \frac{1}{2} w_{\mathrm{win}} \\
		                           0 &   \frac{1}{2} h_{\mathrm{win}} &                                                          0 &                               \frac{1}{2} h_{\mathrm{win}} \\
		                           0 &                              0 & \frac{1}{2} \left(f_{\mathrm{win}}-n_{\mathrm{win}}\right) & \frac{1}{2} \left(f_{\mathrm{win}}+n_{\mathrm{win}}\right) \\
		                           0 &                              0 &                                                          0 &                                                          1
	\end{array} \right),
\end{equation}
where $z$ represents the values for the $z$ buffer.
A formal derivation of this transformation is given in Ref. \cite{web_webgl_viewport}.
% -----------------------------------------------------------------------------



% -----------------------------------------------------------------------------
% rendering / shaders
% -----------------------------------------------------------------------------
\section{Rendering}
\label{sec:gl_shaders}
Rendering in \textit{OpenGL} is performed as pipeline operations comprised of so-called shaders.
Shaders are small programs that are executed on the GPU are are written in the
\textit{OpenGL Shading Language} (\textit{GLSL}) \cite{wiki_glsl}
The pipeline passes buffered vertex data to the vertex shader, performs tessellation steps to triangulate
the resulting geometry, passes this data to a geometry shader, and rasterises it with the help
of a fragment shader \cite[p. 6]{Sellers2002}.
A minimum rendering pipeline to display geometry on the screen consists of at least a vertex and
a fragment shader.
Vertex shaders are called for each vertex of a given geometry and are responsible for the transformations
of the vertices from object into screen space as described above.
Fragment shaders are used for the rasterisation of triangles, i.e. drawing all its pixels on the screen. 
They are, for instance, used to apply colours and textures to polygons.

Constant values, such as the transformation and projection matrices mentioned above, can be propagated from the 
user (CPU) program to the shaders. To that end, \textit{OpenGL} provides API functions that allow the user 
program to write to shader variables that have been declared with the keyword \lstinline[language=C]|uniform|,
see Ref. \cite[pp. 103-126]{Sellers2002}.

The actual geometric information such as vertex positions, vertex normals, but also colours, are declared as
input attributes to the vertex shader using the keyword \lstinline[language=C]|in|.
Attributes can be set from the user program by binding a vertex array object \cite{wiki_vao} to them,
where the array has the exact configuration that is given by the vertex shader's order of \lstinline[language=C]|in|
variables \cite[pp. 97-102]{Sellers2002}.
The elements of the vertex array are each transformed by the vertex shader program, which stores the resulting 
transformed and projected screen space position in a reserved variable named \lstinline[language=C]|gl_Position|.
Further results can be output from the shader by declaring \lstinline[language=C]|out| variables.
These variables are connected to the inputs of the next shader in the rendering pipeline.
% -----------------------------------------------------------------------------
