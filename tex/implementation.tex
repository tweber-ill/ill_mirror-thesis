%
% path-finding
% @author Tobias Weber <tweber@ill.fr>
% @date 2021
% @license see 'LICENSE' file
%

\chapter{Path-finding Details and Implementation}
\label{ch:impl}

In this chapter we describe the individual steps of our strategy (see p. \pageref{sec:strategy}) in detail
and present an implementation in C++ \cite{Stroustrup2008, Stroustrup2018}, specifically 
the latest version 20 of the standard \cite{ISOCPP20}. 
Section \ref{sec:tasmodel} is dedicated to modelling the triple-axis spectrometer (TAS), 
section \ref{sec:buildpath} discusses the steps involved in building up the instrument path, 
and section \ref{sec:exepath} focuses on executing the instrument motion along the path.



% -----------------------------------------------------------------------------
% instrument model
% -----------------------------------------------------------------------------
\section{Modelling the TAS instrument}
\label{sec:tasmodel}


% -----------------------------------------------------------------------------





% -----------------------------------------------------------------------------
% path building
% -----------------------------------------------------------------------------
\section{Building the path}
\label{sec:buildpath}

\subsection{Angular configuration space}


\subsection{Contour tracing}


\subsection{Line-segment generation and simplification}


\subsection{Generation of convex regions}


\subsection{Calculation of the Voronoi diagram}
\label{sec:voronoi}


\subsection{Simplification of the Voronoi diagram}


% -----------------------------------------------------------------------------





% -----------------------------------------------------------------------------
% instrument motion
% -----------------------------------------------------------------------------
\section{Moving the instrument}
\label{sec:exepath}


\subsection{Determination of the start and end coordinates}



\subsection{Calculation of Dijkstra's shortest path}
\label{sec:dijkstra}


% -----------------------------------------------------------------------------
