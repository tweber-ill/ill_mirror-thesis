%
% path-finding
% @author Tobias Weber <tweber@ill.fr>
% @date 2021
% @license see 'LICENSE' file
%

\chapter{Path-finding Details and Implementation}
\label{ch:impl}

In this chapter we describe the individual steps of the strategy (see p. \pageref{sec:strategy}) in detail
and present an implementation in C++ \cite{Stroustrup2008, Stroustrup2018}. Specifically,
the latest version 20 of the C++ standard \cite{ISOCPP20} was employed in the creation of the software
together with the Boost C++ template libraries \cite{web_boost}. The source code for the implementation
can be found in the directory \lstinline|./src/core| together with the library routines in \lstinline|./src/libs|
of the repository at \url{https://code.ill.fr/scientific-software/takin/paths}. Stable versions of the
source code have furthermore been registered under the DOI \href{https://doi.org/10.5281/zenodo.4625649}{10.5281/zenodo.4625649}.

Section \ref{sec:tasmodel} is dedicated to modelling the triple-axis spectrometer (TAS), 
section \ref{sec:buildpath} discusses the steps involved in building up the instrument path, 
and section \ref{sec:exepath} focuses on executing the instrument motion along the path.
The graphical user interface is presented separately, namely in chapter \ref{ch:gui}.





% -----------------------------------------------------------------------------
% instrument model
% -----------------------------------------------------------------------------
\section{TAS instrument modelling}
\label{sec:tasmodel}

The instrument space comprising the triple-axis spectrometer, the walls and obstacles as well as the floor is modelled in
the class \lstinline[language=C++]|InstrumentSpace|. It also serves as high-level interface for loading and saving
the instrument geometry and states, signalling mechanisms for state changes using the publisher-subscribe mechanism
via Boost.Signals2 \cite{web_boost_signals}, as well as checking the instrument for collisions.

The classes \lstinline[language=C++]|Instrument| and \lstinline[language=C++]|Axis| contain the actual instrument definition.
The spectrometer is modelled as a hierarchy of the three principal axes, namely monochromator, sample and analyser.
Each axis has three local coordinate systems, namely the rotation relative to the incoming and outgoing vector, respectively,
and an internal rotation which is decoupled from the other local rotations.
Geometrical objects are derived from the abstract, purely virtual class \lstinline[language=C++]|Geometry| and can be
coupled to any of these three local coordinate systems.
This makes it possible to model neutron-optical components attached to the either the incoming or outgoing path of the
neutron beam at the specific axis. It furthermore makes it possible to have components which rotate independently of
the axis.
The transformation matrices corresponding to the three local coordinate systems are calculated as follows:

\begin{equation}
\begin{split}
	T_{\mathrm{in}} & \ =\  T_{out}^{\mathrm{prev}} \cdot P \cdot R\left(\theta_{\mathrm{in}}\right), \\
	T_{\mathrm{int}} & \ =\  T_{\mathrm{in}} \cdot R\left(\theta_{\mathrm{int}}\right), \\
	T_{\mathrm{out}} & \ =\  T_{\mathrm{in}} \cdot R\left(\theta_{\mathrm{out}}\right).
\end{split}
\end{equation}

Here, $T_{\mathrm{in, int, out}}$ names the transformation matrix of the incoming, internal (decoupled) and outgoing
coordinate system. $T_{out}^{\mathrm{prev}}$ is the outgoing transformation of the preceding instrument axis.
$R\left(\theta_{\mathrm{in, int, out}}\right)$ are the corresponding rotation matrices and $P$ is the translation
of the local coordinate origin of the axis.

% -----------------------------------------------------------------------------





% -----------------------------------------------------------------------------
% path building
% -----------------------------------------------------------------------------
\section{Path-building}
\label{sec:buildpath}

\subsection{Angular configuration space}


\subsection{Contour tracing}


\subsection{Line-segment generation and simplification}


\subsection{Generation of convex regions}


\subsection{Calculation of the Voronoi diagram}
\label{sec:voronoi}


\subsection{Simplification of the Voronoi diagram}


% -----------------------------------------------------------------------------





% -----------------------------------------------------------------------------
% instrument motion
% -----------------------------------------------------------------------------
\section{Instrument movement}
\label{sec:exepath}


\subsection{Determination of the start and end coordinates}



\subsection{Calculation of Dijkstra's shortest path}
\label{sec:dijkstra}


% -----------------------------------------------------------------------------
