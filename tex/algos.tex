%
% gl
% @author Tobias Weber <tweber@ill.fr>
% @date aug-2021
% @license see 'LICENSE' file
%

\chapter{Basic Concepts, Algorithms and Data Structures}
\label{ch:algos}
Before coming to the path-finding algorithm and implementation details, several general concepts, 
data structures and methods, that are employed in this work, are reviewed in a general manner.


\section{Voronoi diagrams}
\label{sec:voro}
As we will see in the next chapter, Voronoi diagrams play a central role for the path-finding 
algorithm of this work. 
A good review of Voronoi diagrams is given in Ref. \cite[Ch. 7, pp. 147-171]{Berg2008} 
and in \cite[Ch. 5, pp. 209f]{FUH_geo2020}, which we follow in this chapter.

Starting with some basic definitions, a \textit{Voronoi diagram} is a set of 
\textit{bisectors}, $B\left(\underline{x},\, \underline{y}\right)$, separating \textit{Voronoi regions}.
A Voronoi region names the set of points $\underline{x}$ in a vectorspace $V$ that are closest to 
a given site $\underline{s}$ under a given metric $\left\Vert \cdot \right\Vert$, which measures distances in $V$.
The site can either be isolated vertices, lines or any other finite object.
Formally, the bisector between two sites $\underline{s}_1$ and $\underline{s}_2$ is, 
generalising from \cite[p. 140]{Icking2001},
\begin{equation}
	B\left(\underline{s}_1,\, \underline{s}_2\right)\ =\ \left\{ \underline{x} \in V \ |\ 
		\left\Vert \underline{x} - \underline{s}_1 \right\Vert = \left\Vert \underline{x} - \underline{s}_2 \right\Vert \right\}.
\end{equation}
Even though this definition of Voronoi diagrams does not restrict the vectorspace to the $\mathbb{R}^n$ 
and the underlying metric to the usual Euclidian one, 
\begin{equation}
	\left\Vert \underline{x} \right\Vert_2 \ =\ \sqrt{\left<x | x \right>},
\end{equation}
they are nevertheless implied in the rest of this work if nothing else is specified.
Specifically, we thus set $V = \mathbb{R}^2$ and $\left\Vert \cdot \right\Vert = \left\Vert \cdot \right\Vert_2$. 
Please refer to \cite{Icking2001} for other metrics.



\subsection{Voronoi diagrams for vertex sites}
The simplest case is the Voronoi diagram for vertex sites.
For the general case, namely $\mathbb{R}^n$, the bisectors consist of $n-1$-dimensional hyperplanes. 
Specifically for $\mathbb{R}^n$, they are either line segments or infinite lines, depending if the 
corresponding Voronoi region is closed or open, respectively.
An example of two or several vertex sites and their bisectors is shown in Fig. \ref{fig:vertex_voro}.

\begin{figure}[htb]
	\begin{minipage}{1 \textwidth}
		\begin{center}
			\includegraphics[width = 0.4 \textwidth]{figures/vertex_voro}
		\end{center}
		\vspace{1cm}
		\begin{center}
			\includegraphics[width = 0.75 \textwidth]{figures/vertex_voro2}
		\end{center}
	\end{minipage}
	\caption[Voronoi diagrams for vertices.]{
		Top panel: Voronoi diagram for two vertex sites, $s_1$ and $s_2$. 
			The bisector, $B\left(s_1, s_2\right)$, separates $\mathbb{R}^2$ 
			into two open Voronoi regions forming half-planes.
		Bottom panel: Voronoi diagram for ten vertex sites, $s_1,\, s_2,\, ...,\, s_{10}$.
		The black lines are the bisectors of the Voronoi regions, where the solid lines are of finite size
		and delimit closed Voronoi regions. The dashed lines are of infinite length and delimit open Voronoi regions.
		The figure has been calculated using the test program that will be described in chapter \ref{sec:tests_hull}.
		\label{fig:vertex_voro}}
\end{figure}



\subsection{Voronoi diagrams for line-segment sites}
\label{sec:voro_ls}
We now look at the special case of Voronoi diagrams constructed from line segments.
A description of this case can be found in \cite[Ch. 7.3, pp. 160-163]{Berg2008} and in 
\cite[pp. 242-247]{FUH_geo2020}, whose descriptions we follow in this section.

For this case, the Voronoi region of a line $l_i$ consists of all points in $\mathbb{R}^2$ that 
are closest to $l_i$. The bisector is the boundary between the Voronoi regions of two line segments $l_i$ and $l_j$, $i \neq j$,
and is the curve of equal distance between the two line segments $l_i$ and $l_j$.
Its shape is either linear or quadratic, where, in the linear case, the bisector curve can also be either
finite or infinite \cite[pp. 243-244]{FUH_geo2020}.

A linear bisector is obtained for the distance calculated between two line segment endpoints or between two inner 
points on the line segments which are not the endpoints.
This can easily be seen, because, (a) in the case of two endpoints, the middle perpendicular line between these 
two points is equidistant to them; and (b) in the case of two line segments, the angular bisector of the two lines is
equidistant to them \cite[pp. 243-244]{FUH_geo2020}.
On the other hand, the bisector curve segment follows a parabolic shape if the distance is calculated 
between a line segment endpoint and one inner point of the other segment. This is the same as the bisector between
a point and a line and it parabolic distance is shown in \cite[pp. 260-261]{FUH_geo2020}.
An example of two or several line segment sites and their bisectors is shown in Fig. \ref{fig:linesegs_voro}.

\begin{figure}[htb]
	\begin{minipage}{1 \textwidth}
		\begin{center}
			\includegraphics[width = 0.75 \textwidth]{figures/linesegs}
		\end{center}
		\vspace{0.25cm}
	\end{minipage}
	\begin{minipage}{1 \textwidth}
		\vspace{0.25cm}
		\begin{center}
			\includegraphics[width = 0.95 \textwidth]{figures/linesegs2}
		\end{center}
	\end{minipage}
	\caption[Voronoi diagrams for line segments.]{
		Top panel: Voronoi regions for two line segments, $l_1$ and $l_2$.
		Bottom panel: Voronoi regions for five line segments, $l_1,\, l_2,\, ...,\, l_5$.
		The line segments and their endpoints are marked in blue. The small red points represent the Voronoi vertices.
		The black lines are the bisectors of the Voronoi regions, where the solid lines delimit finite and the dashed lines
		infinite regions. Helper lines are marked in red. The figure has been calculated using the line segments
		test program which will be described in chapter \ref{sec:tests_linesegs}. The program uses 
		the \textit{Boost.Polygon} library \cite{web_boost_polygon_voronoi}.
		\label{fig:linesegs_voro}}
\end{figure}


\subsection{Software libraries}
A stable and efficient C/C++ library for calculating Voronoi diagram in $\mathbb{R}^n$ is the popular and very 
high-quality \textit{QHull} by C. B. Barber \cite{web_qhull}. While it does only calculate 
the Voronoi diagrams for vertices, and not for line segments, it is also capable of calculation the Delaunay
triangulation and -- as the name implies -- convex hull of the sites.

Several libraries for calculating the line segment Voronoi diagrams in $\mathbb{R}^2$ exist, noteworthy 
are \textit{VRONI} by M. Held \cite{Held2001}, \textit{OpenVoronoi} by A. E. Wallin \cite{web_openvoronoi}, \textit{VoroLS} 
by W. Schumann \cite{DiplomaSchumann}, as well as the Voronoi calculator \cite{web_boost_polygon_voronoi} 
by A. Sydorchukof, which is part of the \textit{Boost.Polygon}  \cite{web_boost_polygon, Simonson2009} 
C++ library.
The first three are feasible for the present project, though: 
\begin{itemize}
	\item \textit{VRONI} \cite{Held2001} is reported in a paper, but is neither freely available 
		nor under a suitable open-source license.
	\item The opposite is true for \textit{OpenVoronoi} \cite{web_openvoronoi}: 
		It is available under an open-source license, but our first tests deemed it too unstable 
		for use in a production-quality software. 
		The source code for our tests can be found in the function \lstinline[language=C++]|geo::calc_voro_ovd()|, 
		which resides in file \lstinline|./src/libs/hull.h| of the accompanying code. 
		The source code for the test tool itself is located in: \lstinline|./src/tools/lines.cpp|.
	\item \textit{VoroLS} \cite{DiplomaSchumann} is stable and does even handle intersecting lines, but is a \textit{Java}
		software, not a library, and is not under an open-source license.
	\item All requirements were met by \textit{Boost.Polygon} \cite{web_boost_polygon}, though, and we will use 
		this library for the calculations of the present work.
		\textit{Boost.Polygon}  uses Fortune's sweep algorithm \cite{Fortune1987} to construct 
		the Voronoi diagram and is internally based on integers coordinates \cite{web_boost_polygon}.
		The reliance on integer coordinate representation is also responsible for \textit{Boost.Polygon}'s 
		inability to handle intersecting lines, because the intersection point may be at a non-integer coordinate. 
		Intersecting lines need to be carefully filtered out, because they can unfortunately cause the 
		\textit{Boost.Polygon} library to either crash, hang, or yield wrong bisectors.
\end{itemize}




\section{Graphs}
\label{sec:graphs}

