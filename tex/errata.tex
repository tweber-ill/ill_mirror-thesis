%
% future errata
% @author Tobias Weber <tweber@ill.fr>
% @date july-2021
% @license see 'LICENSE' file
%

\chapter{Accompanying Software and Future Errata}
\label{ch:online}

\paragraph{Software source code}
The latest version the software that has been developed 
as part of this work has the DOI (digital object identifier)
\href{https://doi.org/10.5281/zenodo.4625649}{10.5281/zenodo.4625649}
and can be found under the URL \url{https://doi.org/10.5281/zenodo.4625649}.
Additionally, the main software's development repository is available here: 
\url{https://code.ill.fr/scientific-software/takin/paths},
a mirror of the repository can be found here: 
\url{https://github.com/tweber-ill/ill_mirror-takin2-paths}.
It has furthermore been archived at this address:
\url{https://archive.softwareheritage.org/browse/origin/?origin_url=https://code.ill.fr/scientific-software/takin/paths}.

The latest version of the software's source code archive (ver-1.0) bears the 
filename ``taspaths-oct21.tar.xz'' and has the SHA-256 \cite{web_sha256sum}
checksum:

\begin{centering}
\texttt{\\
c4addc00595e725c\phantom{.} \\
ff336e31128b29f3\phantom{.} \\
dbb99c05090061aa\phantom{.} \\
2c954115626a42aa.}\\
\end{centering}


\paragraph{Source code layout}
The layout of the software's source code archive is shown in table \ref{tab:sourcelayout}.

\begin{table}[htb]
	\centering
	\begin{tabular}{|c c|c|c|}
		\hline
		\bf{Folder} & & \bf{Explanation} & \bf{References} \tabularnewline
		\hline
		src/ & & Main source directory. & \tabularnewline
		       & libs/ & Library of geometry algorithms and data structures. & Chapter \ref{ch:algos}. \tabularnewline
		       & core/ & Core TAS path finding library. & Chapters \ref{ch:impl} and \ref{sec:library}. \tabularnewline
		       & gui/ & Graphical user interface and 3-D editor/viewer. & Chapter \ref{sec:gui}. \tabularnewline
		       & tools/ & Helper and auxiliary tools. & Chapter \ref{sec:tests_tools}. \tabularnewline
		\hline
		scripting/ & & Python scripting interface. & Chapter \ref{sec:scripting} \tabularnewline
		\hline
		unittests/ & & Unit tests. & Chapter \ref{sec:unit_tests}. \tabularnewline
		\hline
		tests/ & & Small (performance) test programs. & \tabularnewline
		\hline
		res/ & & Resource files, e.g. shader scripts. & \tabularnewline
		\hline
		cmake/ & & Modules for the \textit{CMake} \cite{Martin2007, web_cmake} build system. & \tabularnewline
		\hline
		setup/ & & Compile and setup scripts for... & \tabularnewline
		           & osx/ & ... \textit{MacOS} using \textit{Homebrew} \cite{web_homebrew}, & \tabularnewline
		           & deb/ & ... \textit{GNU/Linux}, mainly \textit{Ubuntu}, & \tabularnewline
		           & mingw/ & ... \textit{MinGW} \cite{web_mingw64}. & \tabularnewline
		\hline
		tlibs2/ & & Mathematical template library developed  & \cite{Takin2016, Takin2017, Takin2021, DiplomaWeber, PhDWeber} \tabularnewline
		           & & during my physics diploma and PhD thesis. & \tabularnewline
		\hline
		externals/ & & Further external library dependencies. & \tabularnewline
		\hline
	\end{tabular}
	\caption[Source code layout.]{Layout of the source code.}
	\label{tab:sourcelayout}
\end{table}

Development of the geometry library under \lstinline|./src/libs| and the auxiliary tools under \lstinline|./src/tools| 
was already started in 2020, as preparation for the exam in Algorithmic Geometry, and their source code 
is therefore based on the course materials \cite{FUH_geo2020}. 
Afterwards, the library and tools have been further developed for this work, their separate code repository is available under 
\url{https://github.com/t-weber/geo}, the software is furthermore registered under the DOI 
\href{https://doi.org/10.5281/zenodo.4297475}{10.5281/zenodo.4297475}.

This software is based on the \textit{tlibs} libraries, which reside in the 
sub-directory \lstinline|./tlibs2|, but are not part of the present thesis.
That library has already been developed since my diploma thesis \cite{DiplomaWeber} 
and my PhD thesis \cite{PhDWeber}, as well as during several software 
projects \cite{Weber2014, Takin2016, Takin2017, Takin2021}.
Its full source code is included in the source archive merely for the convenience of not
having to search the correct version.
Apart from using the \textit{tlibs} library, several code snippets are further developments
of their counterparts from that library, among these the setup and build scripts for the various
systems, as well as the \textit{OpenGL} renderer and general-purpose code.


\paragraph{Thesis source code}
The source of the thesis text can be found in the repository:
\url{https://code.ill.fr/tweber/thesis}, and is mirrored in:
\url{https://github.com/tweber-ill/ill_mirror-thesis}.
The archived copy of the repository can be found here:
\url{https://archive.softwareheritage.org/browse/origin/?origin_url=https://code.ill.fr/tweber/thesis}.


%\vspace{0.75cm}
\paragraph{Errata}
Up-to-date errata for this work will be published via the DOI
\href{https://doi.org/10.5281/zenodo.5092159}{10.5281/zenodo.5092159}
and can be found online under the URL \url{https://doi.org/10.5281/zenodo.5092159}.
