%
% path-finding
% @author Tobias Weber <tweber@ill.fr>
% @date 2021
% @license see 'LICENSE' file
%

This chapter is devoted to the development of the theoretical frameworks for path-finding in a triple-axis spectrometer (TAS). 
The path should furthermore be optimal in the sense that the instrument not only avoids obstacles like walls or equipment in 
the experimental area, but also keeps a maximum distance from them.

Before looking at the situation with TAS in section \ref{sec:tasrobot}, we review the ideas of motion planning for a point-like robot.



\section{Motion planning for a point-like robot}
\label{sec:pointrobot}

The algorithm for motion planning in a point-like robot are given in Ref. \cite[Ch. 13, pp. 283-306]{Berg2008}, whose descriptions 
we follow in this section. While the book chapter also describes polygonal robots, we limit ourselves to the parts of the chapter 
that are relevant for the present work.





\section{Motion planning for a triple-axis spectrometer}
\label{sec:tasrobot}

% TODO: moving instrument along voronoi edges
